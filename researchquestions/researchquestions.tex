\normallinespacing

\chapter{Research Questions} \label{research-questions}
If it is possible to trace the relationship between orchestration load and usability factors, then it would be possible to use this quality as a comparative measure within tools. This thesis will analyse the relationship between usability and orchestration load within PyramidApp for teachers imparting classes online in real time. To do so, it will answer the following questions:
\begin{enumerate}
    \item How do mirroring and guiding dashboard designs affect the orchestration load of a teacher in an online learning situation?
    \item Can changes in orchestration load be traced to usability factors classified as Nielsen's Usability Heuristics?
\end{enumerate}
According to the study design and the conditions that the participants will be exposed to, the hypotheses for each research question are the following. For Research Question 1, the mirroring condition is expected to create a higher orchestration load on the teacher than the alerting condition. The reasoning for this hypothesis is that in the latter, teachers are provided with precise information that helps them in making pedagogical decisions in the moment, while the former does not.\\
The hypothesis for Research Question 2 is that the usability aspects of the teacher-facing dashboard related to Heuristics 1 (Visibility of System Status), 3 (User Control and Freedom) and 9 (Help users recognize, diagnose, and recover from errors) have an observable relationship to cognitive load. The reasoning for this hypothesis is that the mentioned Heuristics account for how the teacher is able to track the time, student performance, and manage extraneous events, which are common things to happen for a teacher in class \cite{Dillenbourg2013-kx}.
\newpage


