\normallinespacing

\chapter{Introduction}
During the past two decades, the adoption of technology has increased exponentially, and with its evolution, it has become present in all aspects of our life. Educational contexts are no exception to this phenomenon, both for teachers and learners \cite{Raja2018-mr}. Teachers are expected to adopt new technology, and to integrate it in their classrooms and into their curriculum \cite{Wasson2020-lq}. Research has shown that technology not always supports teachers, but adds a new layer of complexity to their workload \cite{Prieto2015-gd}. Several tools for teachers have been developed, distinguished as “reflection tools and pedagogical planners, authoring and sharing tools, repositories, and delivery tools” \cite{Persico2015-hp}. Some of these tools are LDShake \cite{Hernandez-Leo2011-qa}, and ScenEdit \cite{Emin2010-dv}.

Technology made possible the development of distance learning, in which teacher and learner are separated in space and/or time \cite{perraton_1981} and where technology serves as the medium through which knowledge is shared \cite{keegan_1988}\cite{garrison_shale_1987}. It broadened the possibilities to engage in formal education with students and teachers around the world. However, under circumstances like those faced throughout 2020, computers in education became not a luxury nor an option, but a need to continue imparting classes. Having gone through these sudden transformations in education, it is reasonable to wonder: How did teachers that are not seasoned in the use of technology manage to convert their classes to a digital format?

Orchestration load has been defined as “the effort necessary for the teacher and other actors-to conduct learning activities” \cite{cuendet_2013} A particular problem in this context is the analysis of usability for teacher-facing tools like dashboards \cite{holden_rada_2011}. When introducing new software to classrooms, understanding their effect on the teachers' workload is critical. This includes having a deep understanding of particular environments for different teachers, defining amount of information being displayed on each tool, and designing the appropriate user interface, as users could get overwhelmed by the amount of information presented to them \cite{Schwendimann2017-ci}. Usability evaluations can provide insights on this topic \cite{Fernandez2011-ln}. Furthermore, the usage of standardised usability evaluations could provide the scientific community an axis on which different educational tools can be compared which, according to Schwendimann et al., \cite{Schwendimann2017-ci} it is still missing.

This project focuses on the usage of PyramidApp, "a technological solution implementing a scalable method applying Pyramid CLFP principles. Individuals propose their option (i.e., can be a question or an answer for a given task, seeking active comprehension) and PyramidApp forms small groups to share ideas about suggested options, to clarify and negotiate before rating the options." \cite{Manathunga2016-gy}. I conducted experiments in which teachers moderated CSCL activities under different conditions and examined how the measurement of involuntary reactions, i.e., Electrodermal Activity (EDA), could be used in triangulation with self-perceptions of the teachers (through post-activity questionnaires) to analyze their orchestration load in each condition. Two conditions were tested. For both conditions, the teacher is presented with the student's progress through the dashboard, and has the same time and level controls over the activity. However, in the "Mirroring" condition the information presented is left for the teacher to interpret on their own, while in the "Guiding" condition alerts are displayed to indicate teachers about critical situations throughout the activity. 